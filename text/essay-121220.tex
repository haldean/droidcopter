%Basic Math or Computer Science Template

\documentclass[letterpaper]{article}
\author{Benjamin Bardin \and William Brown 
  \and Dr. Paul Blaer (Advisor)}
\title{Stable Quadricopter Flight and 
  Telepresence using the Android Platform}

\begin{document}
\maketitle
\tableofcontents
\newpage

\section{Motivation}
\subsection{Objectives}
When we started this project, we had many different use cases for the
helicopter -- automatic target tracking, panaroma creation from the
video stream, even getting it to bring forgotten homework assignments
from our apartment to class. However, for the first iteration of both
the hardware and the software, we set our sights on more
immediately-acheivable goals: telepresence and stable flight
control. For telepresence, we wanted to be able to visualize the
Android device's location, orientation, acceleration and velocity in
real-time, as well as receive a video stream that compensated for
network latency and low bandwidth. For stable flight control, we
wanted a system that could maintain a hovering state within a narrow
radius of a given point (our target radius was ten feet) and could
respond to commands sent from a host computer.

\subsection{Progress}
To date, we've accomplished two major things. We have telepresence
between an Android device and a desktop or laptop computer, using our
Android application and our Java program, and we have an entire stack
of software ready to control and stabilize a
helicopter. Unfortunately, due to unexpected hardware failure (which
will be discussed in section \ref{sec:failure}), we were unable to get
the hardware to a functional state before the end of the
semester. However, the helicopter is fully constructed and, when we
receive the missing part, we will have a helicopter that is ready to
fly.

\subsection{Future Goals}
The first goal is to acheive stable hovering. This will require a fair
amount of debugging, considering how much of our stability and
navigation software remains untested in the field, as well as manual
tuning of PID values (as will be discussed in section
\ref{sec:pilot}), which will no doubt be a lengthy process.

We have also discussed various uses of the helicopter platform we have
created. Currently, we would like to implement blob tracking to allow
the helicopter to track objects as they move. This would allow the
helicopter to track us as we walk around a field, or take a video of
of a skiier, or even to act as a robotic sherpa to follow us while
carrying light objects. We would also like to implement dynamic
panorama creation, in which the helicopter performs a series of
predefined acrobatics to take photos which cover a solid angle of 180
degrees. From this, we can create a panorama from the helicopter's
current location; such birds-eye panoramas would be unusual, if not
unique, and would be both creatively and technically interesting to
generate.

\section{Hardware}
\subsection{Design of the Chassis}
The chassis was designed to be as modular as possible. 

\subsection{Hardware Selection}

\subsection{Hardware Usage}

\subsection{Hardware Failure}
\label{sec:failure}

\section{Pilot Android Application}
\label{sec:pilot}

\section{Server Software}

\section{Communication}

\end{document}